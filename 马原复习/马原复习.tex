\documentclass[UTF8]{ctexart}
\title{马原复习}
\author{treasurew}
\date{\today}
\usepackage[hidelinks]{hyperref}

\begin{document}
	\tableofcontents
	\maketitle
	\section{简答题}
		\subsection{简述马克思主义的三个组成部分及其相互作用关系?(P2)}
			\paragraph{三个组成部分}
				马克思主义哲学、马克思主义政治经济学、科学社会主义。
			\paragraph{相互作用关系}
				它们有机统一并构成了马克思主义理论的主体内容。
		\subsection{哲学的基本问题是什么?(P20-21)}
			存在和思维的关系问题又称为物质和精神的关系问题,构成了全部哲学的基本问题。\par
			存在和思维的关系问题包括两个方面的内容:第一性问题(存在和思维谁是世界的本原)和同一性问题(思维能否正确认识存在)。
		\subsection{简述意识对物质的反作用。(P26)}
			\paragraph{第一}
				意识活动具有目的性和计划性。
			\paragraph{第二}
				意识活动具有创造性。
			\paragraph{第三}
				意识具有指导实践改造客观世界的作用。
			\paragraph{第四}
				意识具有调控人的行为和生理活动的作用。
		\subsection{简述如何处理主观能动性和客观规律性的关系。(p27-28)}
			\paragraph{一方面}
				尊重客观规律是正确发挥主观能动性的前提。\par
			\paragraph{另一方面}
				只有充分发挥主观能动性,才能正确认识和利用客观规律。\par
			\paragraph{第一}
				从实际出发是正确发挥人的主观能动性的前提。\par
			\paragraph{第二}
				实践是正确发挥人的主观能动性的基本途径。\par
			\paragraph{第三}
				正确发挥人的主观能动性,还需要依赖于一定的物质条件和物质手段。\par
			在社会历史领域,主观能动性与客观规律的辩证关系具体表现为社会历史趋向与主体选择的关系。社会历史规律的客观性和必然性规定了人的活动要受规律性的制约,但与此同时,又不能否定人作为历史主体的能动性和选择性。\par
			坚持发展中国特色社会主义道路体现了二者的辩证统一。
		\subsection{为什么说对立统一规律是唯物辩证法的实质和核心?(P37)}
			对立统一规律揭示了事物普遍联系的根本内容和变化发展的内在动力,从根本上回答了事物为什么会发展的问题。\par
			对立统一规律是贯穿量质变规律、否定之否定规律以及唯物辩证法基本范畴的中心线索,也是理解这些规律和范畴的“钥匙”。\par
			对立统一规律提供了人们认识世界和改造世界的根本方法——矛盾分析方法。
		\subsection{简述量变与质变的辩证关系。(P40)}
			\paragraph{第一}
				量变是质变的必要准备。任何事物的变化都有一个量变的积累过程,没有量变的积累,质变就不会发生。\par
			\paragraph{第二}
				质变是量变的必然结果。量变到一定程度必然引起质变。\par
			\paragraph{第三}
				量变和质变是相互渗透的。一方面,在总的量变过程中有阶段性和局部性的部分质变;另一方面,在质变过程中也有旧质在量上的收缩和新质在量上的扩张。\par
			量变和质变是相互依存、相互贯通的,量变引起质变,在新质的基础上,事物又开始新的量变,如此交替循环,构成了事物的发展过程。量变质变规律体现了事物发展的渐进性和飞跃性的统一。
		\subsection{简述如何坚持对立统一规律“两点论”和“重点论”的统一?(P39)}
			\paragraph{“两点论”}
				是指在分析事物的矛盾时,不仅要看到矛盾双方的对立,而且要看到矛盾双方的统一;\par
				不仅要看到矛盾体系中存在着主要矛盾、矛盾的主要方面,而且要看到次要矛盾、矛盾的次要方面。
			\paragraph{“重点论”}
				是指要着重把握主要矛盾、矛盾的主要方面,并以此作为解决问题的出发点。\par
				“两点论”和“重点论”的统一要求我们,看问题既要全面地看,又要看主流、大势、发展趋势。
		\subsection{简述实践在认识活动中的决定作用?(P61-63)}
			\paragraph{第一}
				实践是认识的来源。认识的内容是在实践活动的基础上产生和发展的。人们只有通过实践实际地改造和变革对象,才能准确把握对象的属性、本质和规律,形成正确的认识,并以这种认识指导人的实践活动。
			\paragraph{第二}
				实践是认识发展的动力。实践的需要推动认识的产生和发展,推动人类的科学发现和技术发明,推动人类的思想进步和理论创新。
			\paragraph{第三}
				实践是认识的目的。人们通过实践获得认识,最终目的是为实践服务,指导实践,以满足人们生活和生产的需要。
			\paragraph{第四}
				实践是检验认识真理性的唯一标准。认识是否具有真理性,既不能从认识本身得到证实,也不能从认识对象中得到回答,只有在实践中才能得到验证。
		\subsection{简述感性认识和理性认识的辩证统一关系?(P68-69)}
			\paragraph{第一}
				感性认识有待于发展和深化为理性认识。
			\paragraph{第二}
				理性认识依赖于感性认识。
			\paragraph{第三}
				感性认识和理性认识相互渗透、相互包含。
		\subsection{简述实践与认识的辨证运动及规律。(P72)}
			实践与认识的辩证运动,是一个由感性认识到理性认识,又由理性认识到实践的飞跃,是实践、认识、再实践、再认识,循环往复以至无穷的辩证发展过程。\par
			这个过程既不是封闭式的循环,也不是直线式的发展,往往充满了曲折以至反复,因而是一个波浪式前进和螺旋式上升的过程。\par
			在实践和认识的辩证运动中,主观必须统一于客观,认识必须统一于实践。这种统一是认识和实践的矛盾在发展中的统一,是具体的历史的统一。即“主观和客观、理论和实践、知和行的具体的历史的统一”。
		\subsection{简述价值的基本特征。(P86-88)}
			价值具有主体性、客观性、多维性和社会历史性四个基本特征,它们是价值本质的体现。
			\paragraph{主体性}
				价值的主体性是指价值直接同主体相联系,始终以主体为中心。\par
				其一,价值关系的形成依赖于主体的存在。\par
				其二,价值关系的形成依赖于主体的创造,使客体潜在的价值转化为现实的存在。
			\paragraph{客观性}
				价值的客观性是指在一定条件下客体对于主体的意义不依赖于主体的主观意识而存在。
			\paragraph{多维性}
				价值的多维性是指每个主体的价值关系具有多样性,同一客体相对于主体的不同需要会产生不同的价值。
			\paragraph{社会历史性}
				主体和客体的不断变化决定了价值的社会历史性特点。人类社会历史发展决定了价值的社会历史性。
		\subsection{为什么物质资料生产方式是社会历史发展的\\决定力量?(P108)}
			物质资料生产方式是指人们为获取物质生活资料而进行的生产活动方式,它是生产力和生产关系的统一体。
			\paragraph{首先}
				物质生产活动条件中,生产方式是人类赖以存在和发展的基础,是人类其他一切活动的首要前提。
			\paragraph{其次}
				物质生产活动及生产方式决定着社会的结构、性质和面貌,制约着人们的经济生活、政治生活和精神生活等全部社会生活。
			\paragraph{最后}
				物质生产活动及生产方式的变化发展决定整个社会历史的变化发展,决定社会形态从低级向高级的更替和发展。
		\subsection{简述社会意识的相对独立性的具体表现。(P112-113)}
			社会意识在从根本上受到社会存在决定的同时,还具有自己特有的发展形式和规律。
			\paragraph{第一}
				社会意识与社会存在发展的不完全同步性和不平衡性。
			\paragraph{第二}
				社会意识内部各种形式之间的相互影响及各自具有的历史继承性。
			\paragraph{第三}
				社会意识对社会存在具有能动的反作用。这是社会意识相对独立的突出表现。
		\subsection{简述生产关系一定要适合生产力状况的规律。(P118-120)}
			生产力和生产关系是社会生产不可分割的两个方面。在社会生产中,生产力是生产的物质内容,生产关系是生产的社会形式。二者的有机结合统一构成社会的生产方式。
			\paragraph{第一}
				生产力决定生产关系。其一,生产力状况决定生产关系的性质。其二,生产力的发展决定生产关系的变化。
			\paragraph{第二}
				生产关系对生产力具有能动的反作用。
			\paragraph{就内容看}
				这一规律概括了生产力和生产关系相互作用的两个方面。
			\paragraph{从过程看}
				这一规律表现为生产关系对于生产力总是从基本相适合到基本不相适合,再到新的基础上的基本相适合;与此相适应,生产关系也总是从相对稳定到新旧更替,再到相对稳定。生产力和生产关系的这种矛盾运动循环往复,不断推动社会生产发展,进而推动整个社会逐步走向高级阶段。生产关系一定要适合生产力状况的规律是社会形态发展的普遍规律。
		\subsection{社会基本矛盾在历史发展中的作用主要表现在\\哪些方面?(P131-132)}
			社会基本矛盾是历史发展的根本动力。
			\paragraph{首先}
				生产力是社会基本矛盾运动中最基本的动力因素,是人类社会发展和进步的最终决定力量。\par
				生产力是社会进步的根本内容,是衡量社会进步的根本尺度。
			\paragraph{其次}
				社会基本矛盾特别是生产力和生产关系的矛盾,决定着社会中其它矛盾的存在和发展。\par
				经济基础和上层建筑的矛盾也会影响和制约生产力和生产关系的矛盾。
			\paragraph{最后}
				社会基本矛盾具有不同的表现形式和解决方式,并从根本上影响和促进社会形态的变化和发展。
		\subsection{简述科学技术在社会发展中的作用。(P144-147)}
			\paragraph{首先}
				对生产方式产生了深刻影响。其一,改变了社会生产力的构成要素。其二,改变了人们的劳动形式。其三,改变了社会经济结构,特别是导致产业结构发生变革。
			\paragraph{其次}
				对生活方式产生了巨大影响。
			\paragraph{最后}
				促进了思维方式的变革
			\paragraph{总之}
				科学技术是社会发展的重要动力。
			\paragraph{消极影响}
				科学技术的作用既受到一定客观条件如社会制度、利益关系的影响,也受到一定主观条件如人们的观念和认识水平的影响。\par
				对于自然规律和人与自然的关系认识不够,或缺乏对科学技术消极后果的强有力的控制手段。\par
				在资本主义条件下,科学技术常常被资产阶级用作剥削压迫人民的工具。
		\subsection{简述商品的二因素之间的对立统一关系(P162-163)}
			\paragraph{对立性}
				商品的使用价值和价值是相互排斥的,二者不可兼得。
			\paragraph{统一性}
				作为商品,必须同时具有使用价值和价值两个因素。
		\subsection{简述人民群众在创造历史过程中的决定作用?(P150-153)}
			人民群众实例社会历史的主体,是历史的创造者。\par
			在社会历史发展过程中,人民群众起着决定性的作用。\par
			人民群众是社会物质财富的创造者。\par
			人民群众是社会精神财富的创造者。\par
			人民群众是社会变革的决定力量。\par
		\subsection{简述价值规律及其作用。(P166-169)}
			\paragraph{价值规律}
				价值规律是商品生产和商品交换的基本规律。\par
				商品的价值量由生产商品的社会必要劳动时间决定,商品交换以价值量为基础,按照等价交换的原则进行。\par
				价值规律贯穿于商品经济的全部过程,它既支配商品生产,又支配商品流通。\par
				在商品经济中,价值规律的表现形式是,商品的价格围绕商品的价值自发波动。\par
			\paragraph{作用}
				价值规律是在市场配置资源的过程中体现它的客观要求和作用的。\par
				第一,自发地调节生产资料和劳动力在社会各生产部门之间的分配比例。\par
				第二,自发地刺激社会生产力的发展。\par
				第三,自发地调节社会收入的分配。\par
				消极后果:第一,导致社会资源浪费;第二,阻碍技术进步;第三,导致收入两极分化。
		\subsection{简述经济全球化的表现。(P228-230)}
			\paragraph{第一}
				国际分工进一步深化。
			\paragraph{第二}
				贸易全球化。
			\paragraph{第三}
				金融全球化。
			\paragraph{第四}
				企业生产经营全球化。
		\subsection{简述什么是辩证否定观?(P41)}
			\paragraph{第一}
				否定是事物的自我否定,是事物内部矛盾运动的结果。
			\paragraph{第二}
				否定是事物发展的环节,是旧事物向新事物的转变,是从旧质到新质的飞跃。只有经过否定,旧事物才能向新事物转变。
			\paragraph{第三}
				否定是新旧事物联系的环节,新事物孕育产生于旧事物,新旧事物是通过否定环节联系起来的。
			\paragraph{第四}
				辩证否定的实质是“扬弃”,即新事物对旧事物既批判又继承,既克服其消极因素又保留其积极因素。
		\subsection{简述劳动二重性与商品二因素之间的关系。P162-163)}
			劳动二重性决定商品的二因素\par
			\paragraph{商品二因素}
				商品的二因素是价值和使用价值。\par
				商品的使用价值和价值是相互排斥的,二者不可兼得。\par
				作为商品,必须同时具有使用价值和价值两个因素。\par
			\paragraph{劳动二重性}
				商品是劳动产品,生产商品的劳动可区分为具体劳动和抽象劳动。\par
				具体劳动是指生产一定使用价值的具体形式的劳动。\par
				抽象劳动是指撇开一切具体形式的、无差别的一般人类劳动,即人的脑力劳动和体力的耗费。\par
			\paragraph{二者之间的关系}
				生产商品的具体劳动创造商品的使用价值,抽象劳动形成商品的价值。具体劳动和抽象劳动是同一劳动的两种规定。任何一种劳动,一方面是特殊的具体劳动,另一方面又是一般的抽象劳动,这就是劳动的二重性。正是劳动的二重性决定了商品的二因素。
\end{document}